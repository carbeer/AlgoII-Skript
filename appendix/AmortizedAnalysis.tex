\chapter{Mathematische Grundlagen}
\section{Amortisierte Analyse}

\term{Amortisierte Laufzeiten} stellen eine Worst-Case-Analyse für eine Sequenz von Operationen dar.

$A_{Op}$ sind die amortisierten Kosten einer Operation $Op \in \mathcal{O}$.
$T_{Op}$ sind die tatsächlichen Kosten einer Operation $Op \in \mathcal{O}$.

Die amortisierten Kosten sind korrekt, wenn für konstantes $c$ gilt:

\begin{equation}
	\sum_{1\leq i \leq n}T_{Op_i} \leq c + \sum_{1 \leq i \leq n}A_{Op_i}
\end{equation}

\subsection{Bankkonto- oder Potentialmethode}
Finde ein Gehalt pro Operation, welches eine Obergrenze für die Kosten der Sequenz der Operationen darstellt. 

\subsection{Aggregatmethode}
Schätze $\sum_{1 \leq i \leq n}{T_{Op}^i}$ direkt ab.

\subsection{Pseudotree}
Ein zusammenhängender Pseudoforest, d.g. ein Graph der maximal einen Kreis enthält. Vom Kreis ausgehend sind es Trees.

\section{Komplexitätsklassen}
 
\begin{itemize}
	\item $P$ 
	\item $NP$
	\item $NP\text{-vollständig}$: Menge der Probleme die in $NP$ liegen und $NP\text{-schwer}$ sind.
	\item $NP\text{-schwer}$ oder $NP\text{-hart}$: Menge der Probleme auf die sich $NP\text{-Probleme}$ reduzieren lassen.
\end{itemize}

\section{Hyperwürfel}
Ein Hyperwürfel der Dimension $n$ entsteht durch Verbindung zweier $n-1$-dimensionaler Würfel, wobei die äquivalenten Knoten jeweils miteinander verbunden werden. EIn Hyperwürfel in Dimension 0 ist ein einziger Knoten.