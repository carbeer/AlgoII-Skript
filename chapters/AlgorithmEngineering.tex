\chapter{Algorithm Engineering}
Algorithm Engineering ist ein interdisziplinäres Forschungsgeld welches Lücken zwischen Theorie und Praxis schließt. Es umfasst Theorie, Praxis und Anwendung. Theoretische Algorithmik ist ein Unterfeld des Algorithm Engineering.

Die Aufgaben lassen sich unterteilen in:
\begin{itemize}
  \item \textbf{Modellierung} 
  \item \textbf{Design}: Einfachheit und Wiederverwendbarkeit 
  \item \textbf{Analysis}: Berücksichtigung konstanter Faktoren und durchschnittlicher Laufzeiten
  \item \textbf{Implementierung}
  \item \textbf{Experimente}: Reproduzierbarkeit wichtig. Software Engineering wird miteinbezogen. Können bei der Analyse helfen. 
\end{itemize}

Im Gegensatz zu rein theoretischer Algorithmik herrscht eine größere Methodevielfalt und größerer Bezug zu Anwendungen. 
