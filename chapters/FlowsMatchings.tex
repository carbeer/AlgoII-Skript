\chapter{Maximale Flüsse und Matchings}

% helpers
\newcommand{\pluseq}{\mathrel{+}=}
\newcommand{\minuseq}{\mathrel{-}=}

\section{Definitionen}

\subsection{Netzwerk}

Ein \term{Netzwerk}\index{Netzwerk} ist ein gerichteter und gewichteter Graph mit zwei speziellen Knoten --- einer \term{Quelle}\index{Netzwerk!Quelle} und einer \term{Senke}\index{Netzwerk!Senke}.

Unterschied zwischen \( s \), \( t \) und dem Rest der Knoten ist, dass \( s \) keine eingehenden und \( t \) keine ausgehenden Kanten hat. 

Das Gewicht \( c_e \) der Kante \( e \) nennen wir die \term{Kapazität}\index{Netzwerk!Kapazität} der Kante. Diese muss nicht-negativ sein.

\begin{figure}[H]
  \includegraphics[width=0.6\textwidth]{network}
  \caption{Beispiel für ein Netzwerk mit Quelle \( s \) (\emph{source}) und Senke \( t \) (\emph{sink})}
\end{figure}

\subsection{Fluss}

Ein \term{Fluss}\index{Fluss} in einem Netzwerk ist eine Funktion \( f_e \) auf den Kanten des Netzwerks. Ein Fluss hat folgende Eigenschaften:

\begin{itemize}
  \item \( 0 \leq f_e \leq c_e \) (\( \forall e \in E \))
  \item \( \forall v \in V \setminus \left \{ s,t \right \} : \) Summe eingehender Flüsse = Summe ausgehender Flüsse
\end{itemize}

Wir definieren außerdem den \term{Wert}\index{Fluss!Wert} eines Flusses \( f \) als
\begin{equation*}
  \text{val}(f) = \Sigma \text{ von \( s \) ausgehender Fluss} = \Sigma \text{ zu \( t \) eingehender Fluss}
\end{equation*}

Ziel ist es in der Regel, einen Fluss in einem festgelegten Netzwerk mit \emph{maximalem Wert} zu finden.

\section{Cuts}

\begin{definition}[Cut]
  Ein \( s \)-\( t \)-\term{Cut}\index{Cut} ist eine Partitionierung eines Graphen \( G \) in zwei Mengen \( S \) und \( T \), sodass \( s \in S \) und \( t \in T \).

  Die \term{Kapazität}\index{Cut!Kapazität} eines Cuts ist
  \begin{equation*}
    \sum\left \{ c_{(u,v)} : u \in S \wedge v \in T \right \}\text{,}
  \end{equation*}
  also die Summe der Kapazitäten aller Kanten, die durch den Cut ``durchgeschnitten'' werden.
\end{definition}

\begin{figure}[H]
  \includegraphics[width=0.6\textwidth]{flow}
  \caption{Die blauen Zahlen stellen einen möglichen Fluss in obigem Netzwerk dar. Die schwarzen Kanten sind ein möglicher Cut. Der Wert dieses Flusses ist 18}
\end{figure}

Es gilt folgender extrem praktischer Satz:

\begin{theorem}[Fluss-Cut-Dualität]
  Der Wert eines maximalen \( s \)-\( t \)-Flusses ist die minimale Kapazität eines \( s \)-\( t \)-Cuts.
\end{theorem}

Wie können wir nun diesen Satz nutzen, um einen maximalen Fluss zu finden? Eine Möglichkeit ist \emph{Lineare Programmierung}, aber es gibt bessere Lösungen.

\section{Pfade augmentieren}

Idee ist folgende:
\begin{enumerate}
  \item Wähle einen \( s \)-\( t \)-Pfad, der noch Kapazität übrig hat.
  \item Sättige die Pfadkante mit der kleinsten Restkapazität.
  \item Korrigiere die Kapazitäten aller anderen Kanten mithilfe des \emph{Residualgraphen}. Gehe wieder zu Schritt (1) und wiederhole den Prozess.
\end{enumerate}

\subsection{Residualgraph}

Ist ein Netzwerk \( G = (V, E, c) \) mit Fluss \( f \) gegeben, so erhalten wir den \term{Residualgraphen}\index{Residualgraph} \( G_f = (V, E_f, c^f) \). Dabei gilt für jedes \( e \in E \):
\begin{equation*}
  \begin{cases}
    e \in E_f \text{ mit } c_e^f = c_e - f(e) \quad &\text{falls } f(e) < c(e)  \\
    e^\text{rev} \in E_f \text{ mit } c_{e^\text{rev}}^f = f(e) \quad &\text{falls } f(e) > 0
  \end{cases}
\end{equation*}
Dabei ist für \( e = (u,v) \in E \) die Kante \( e^\text{rev} = (v,u) \).

Die Kanten in die ``normale'' Richtung sind also diejenigen, die noch Restkapazität haben; die neue Kapazität ist die Restkapazität.

Die Kanten in umgekehrte Richtung sind die Kanten, wo der Fluss \( > 0 \) ist (das heißt insbesondere, dass auch beide Fälle eintreten können). Das Gewicht dieser Kanten entspricht dem Fluss der Kanten in normale Richtung.

\begin{minipage}{.475\textwidth}
  \begin{figure}[H]
    \includegraphics[width=\textwidth]{flow2}
    \caption{Nochmal der Fluss von oben}
  \end{figure}
\end{minipage}
\hfill
\begin{minipage}{.475\textwidth}
  \vspace{6.8mm}
  \begin{figure}[H]
    \includegraphics[width=\textwidth]{residualFlow}
    \caption{Der zu nebenstehendem Fluss gehörende Residualgraph. Schwarz die Kapazität, Blau der Fluss, Rot die residuale Kapazität}
  \end{figure}
\end{minipage}

Wir suchen jetzt nach einem \( s \)-\( t \)-Pfad \( p \), sodass jede Kante residuale Kapazität \( c_e^f \neq 0 \) hat:

\begin{pseudocode}
  \( \Delta f \coloneqq \min_{e \in p}c_e^f \) \\
  \textbf{foreach} \( (u,v) \in p \) \textbf{do} \\
  \phantom{\enskip} \textbf{if} \( (u,v) \in E \) \textbf{then} \( f_{(u,v)} \pluseq \Delta f \) \enskip{} \textcolor{gray}{// forward edge} \\
  \phantom{\enskip} \textbf{else} \( f{(v,u)} \minuseq \Delta f \) \enskip{} \textcolor{gray}{// backward edge} 
\end{pseudocode}

Wir können nun den \term{Ford-Fulkerson-Algorithmus}\index{Ford-Fulkerson-Algorithmus} implementieren:

\begin{pseudocode}
  \textbf{\textsc{FFMaxFlow}}\( (G = (V,E), s, t, c: E \to \N) : E \to \N \) \\
  \phantom{\enskip} \( f \coloneqq 0 \) \\
  \phantom{\enskip} \textbf{while} \( \exists \) path \( p = (s,\dots,t) \) in \( G_f \) \textbf{do} \\
  \phantom{\enskip} \phantom{\enskip} augment \( f \) along \( p \) \\
  \phantom{\enskip} \textbf{return} f
\end{pseudocode}

Die Zeitkomplexität ist (bei terminierendem FF) \( O(m*\text{val}(f)) \), da bei jedem Durchgang der Fluss um mindestens \( 1 \) erhöht wird und jedes Mal DFS \( O(m) \) ausgeführt werden muss. Es gibt Netzwerke, in denen der Ford-Fulkerson-Algorithmus tatsächlich die längstmögliche Laufzeit braucht, obwohl die Lösung sehr trivial ist. 

The maximum flow equals the minimum cut capacity.

\subsection{Problem --- Blocking Flows}

\begin{minipage}{.6\textwidth}
  
  In einigen Fällen konvergiert der Algorithmus nicht. Grund dafür sind sogenannte \term{Blocking Flows}\index{Fluss!Blocking Flow}.

  \( f_b \) ist ein \emph{blocking flow} in \( H \), falls für jeden \( s \)-\( t \)-Pfad \( p \) gilt:
  \begin{equation*}
    \exists \ e \in p : f_b(e) = c(e)\text{.}
  \end{equation*}
  Das bedeutet, dass jeder \( s \)-\( t \)-Pfad eine vollständig ausgelastete Kante beinhaltet.
\end{minipage}
\hfill
\begin{minipage}{.35\textwidth}
  \begin{figure}[H]
    \includegraphics[width=\textwidth]{fulkersonBad}
    \caption{In diesem Netzwerk könnte der Ford-Fulkerson-Algorithmus stets einen Pfad über die mittlere Kante augmentieren und würde deswegen 200 Schritte brauchen}
  \end{figure}
\end{minipage}

\section{Dinic-Algorithmus}

Der \term{Dinic-Algorithmus}\index{Dinic-Algorithmus} sieht so aus:

\begin{pseudocode}
  \textbf{\textsc{DinitzMaxFlow}}\( (G = (V, E), s, t, c: E \to \N) : E \to \N \) \\
  \phantom{\enskip} \( f \coloneqq 0 \) \\
  \phantom{\enskip} \textbf{while} \( \exists \) path \( p = (s,\dots,t) \) in \( G_f \) \textbf{do} \enskip{} \textcolor{gray}{// an augmented edge exists}\\
  \phantom{\enskip} \phantom{\enskip} \( d = G_f\text{.\textcolor{red}{reverseBFS}}(t) : V \to \N \) \enskip{} \textcolor{gray}{// find shortest augmenting paths (distance) to source}\\
  \phantom{\enskip} \phantom{\enskip} \( \textcolor{red}{L_f} \coloneqq (V, \left \{ (u,v) \in E_f : d(v) = d(u) - 1 \right \}) \) \enskip{} \textcolor{gray}{// build layer graph} \\
  \phantom{\enskip} \phantom{\enskip} \textcolor{red}{find blocking flow} \( f_b \) in \( L_f \) \enskip{} \textcolor{gray}{// with DFS from source on layer graph}\\
  \phantom{\enskip} \phantom{\enskip} augment \( f \pluseq f_b \) \\
  \phantom{\enskip} \textbf{return} \( f \)
\end{pseudocode}

Die rot gekennzeichneten Funktionen/Objekte müssen wir noch klären.

Der Ablauf lässt sich folgendermaßen zusammenfassen:

\begin{enumerate}
  \item Berechne Distanz-Labels (Abstand zur Senke) für alle Knoten des Graphen.

   (Rückwärtsgerichtete Breitensuche von \( t \) aus)
  \item Stelle auf Basis der Distanz-Labels den Layer-Graph auf.
  \item Suche auf dem Layer-Graph einen blockierenden Fluss zwischen \( s \) und \( t \).
  \item Führe den gefundenen blockierenden Fluss auf dem Residualgraph aus und aktualisiere diesen entsprechend.
  \item Konstruiere den aktualisierten Layer-Graphen und gehe zu Schritt 3.
  \item Breche ab, sobald kein weiterer Fluss im Layer-Graph existiert.
\end{enumerate}

\subsection{Abstandsfunktion}

Die Abstandsfunktion \( d \) gibt den Abstand eines Knotens zur Senke \( t \) an.

\subsection{Layer-Graph}

Den \term{Layer-Graph}\index{Layer-Graph} eines Netzwerks erhält man, indem man alle Kanten aus dem Residualgraphen entfernt, die nicht von einer Schicht in die vorherige führen. Es werden also alle Kanten
\begin{itemize}
  \item innerhalb einer Schicht und
  \item zwischen Schicht \( i \) und \( i + k \) (\( k \in \N \))
\end{itemize}
entfernt. Formal ist das
\begin{equation*}
  L_f = (V, \left \{ (u,v) \in E_f : d(v) = d(u) - 1 \right \})\text{.}
\end{equation*}

\subsection{Laufzeitanalyse}

Blocking Flows ergibt sich mittels 

\begin{equation*}
	n \cdot \text{\#breakthroughs} + \text{\#extends} + \text{\#retreats}
\end{equation*}

eine Laufzeit von $O(m + nm) = O(nm)$.
Spezialfälle mit niedrigerer Komplexität sind Unit Flows (mit Einheitskantengewichten) und Dynamic Trees. Letztere verringern die Komplexität von Breakthroughs, erhöhen sie aber für Retreats und Extends. Dadurch sind in der Praxis in der Regel schlechtere Ergbnisse verbunden.

Die Laufzeit des Dinic-Algorithmus ist damit bei maximal $n$ Durchgängen \( O(mn^2) \) (stark polynomiell). Bei Verwendung von Dynamic Trees käme man auf $O(mn \cdot \log n)$.

Unit Trees lassen sich in $O((m+n)\sqrt{m})$ berechnen. Schränkt man Unit Trees weiter ein, und fordert dass alle Kanten $v$ die Gleichung $\min{\text{indegree}(v), \text{outdegree}(v)} = 1$ erfüllen, so sprechen wir von einem Unit Network. Dieses ist durch $O(m+n\sqrt{n})$ beschränkt.


\section{Matchings}

Ein \term{Matching}\index{Matching} in einem ungerichteten Graphen \( G = (V, E) \) ist eine Teilmenge der Kanten, sodass es keine Kanten gibt, die einen gemeinsamen Knoten berühren.

Ein Matching ist \emph{maximal}, wenn es keine Kante gibt, die man aus \( E \) zum Matching hinzufügen könnte, ohne die Matching-Anforderung zu verletzen.

Ein Matching hat \emph{maximale Kardinalität}, wenn es kein Matching auf \( G \) gibt, das mehr Knoten abdeckt.

Wir werden Matchings zuerst einmal verwenden, um Flüsse zu berechnen. Dazu werden Source und Sink in den Graphen eingefügt und der Dinic-Algorithmus verwendet. Da dies ein Unit Flow Network ist, funktioniert dies in $O((n+m)\sqrt{n})$.

Eine Verallgemeinerung für gewichtete Graphen ist nur sehr begrenzt möglich, indem eine Kontraktion von SCCs mit hoher Kapazität vorgenommen wird. 

\section{Preflow-Push}

Nachteil von pfadaugmentierenden Algorithmen ist, dass man Pfade mit hoher Kapazität aufgrund von späteren Kanten mit geringer Kapazität sehr häufig durchgehen muss.

\term{Preflow-Push-Algorithmen}\index{Preflow-Push-Algorithmus} lösen dieses Problem.

\begin{definition}[Preflow]
  Ein \term{Preflow}\index{Preflow} ist ein Fluss \( f \), bei dem die Summe der eingehenden Flüsse für einen Knoten höher sein darf als die Summe der ausgehenden Flüsse. Die Differenz dieser Summen nennen wir den \term{Exzess}\index{Exzess} des Knotens:
  \begin{equation*}
    \text{excess}(v) \coloneqq \underbrace{\sum_{(u,v) \in E} f_{(u,v)}}_{\text{inflow}} - \underbrace{\sum_{(v,w) \in E} f_{(v,w)}}_{\text{outflow}} \geq 0\text{.}
  \end{equation*}
  Wir nennen einen Knoten \( v \in V \setminus \left \{ s, t \right \} \) \term{aktiv}, falls \( \text{excess}(v) > 0 \).
\end{definition}

Wir definieren nun die \code{push}\index{Push}-Funktion:

\begin{pseudocode}
  \textbf{\textsc{push}}\( (e = (v,w), \delta) \) \\
  \phantom{\enskip} \textbf{assert} \( \delta > 0 \) \\
  \phantom{\enskip} \textbf{assert} \(  \text{excess}(v) \geq \delta \) \enskip{} \textcolor{gray}{// node is active}\\
  \phantom{\enskip} \textbf{assert} residual capacity of \( e \geq \delta \) \enskip{} \textcolor{gray}{// edge can handle flow increase}\\
  \phantom{\enskip} \( \text{excess}(v) \minuseq \delta \) \\
  \phantom{\enskip} \( \text{excess}(w) \pluseq \delta \) \\
  \phantom{\enskip} \textbf{if} \( e \) is reverse edge \textbf{then} \( f(\text{reverse}(e)) \minuseq \delta \) \\
  \phantom{\enskip} \textbf{else} \( f(e) \pluseq \delta \)
\end{pseudocode}

Wir nennen einen Push
\begin{itemize}
  \item \term{saturierend}\index{Push!saturierend}, falls \( \delta = c_e^f \),
  \item \term{nicht-saturierend}\index{Push!nicht saturierend}, falls \( \delta < c_e^f \).
\end{itemize}

\subsection{Level-Funktion}

Idee ist nun, von \( s \) aus Richtung \( t \) zu pushen. Dazu benötigen wir eine Approximation \( d(v) \) der BFS-Distanz von \( v \) zu \( t \) in \( G_f \).

Wir können nun den Preflow-Push-Algorithmus implementieren:

\begin{pseudocode}
  \textbf{\textsc{genericPreflowPush}}\( (G = (V, E), f) \) \\
  \phantom{\enskip} \textbf{forall} \( e = (s,v) \in E \) \textbf{do} \( \text{push}(e, c(e)) \) \\
  \phantom{\enskip} \( d(s) \coloneqq n \), \( d(v) = 0 \) for all other nodes \\
  \phantom{\enskip} \textbf{while} \( \exists \ v \in V \setminus \left \{ s, t \right \} : \text{excess}(v) > 0 \) \textbf{do} \\
  \phantom{\enskip} \phantom{\enskip} \textbf{if} \( \exists \ e = (v,w) \in E_f : d(w) < d(v) \) \textbf{then} \enskip \textcolor{gray}{// eligible (decending) edge} \\
  \phantom{\enskip} \phantom{\enskip} \phantom{\enskip} choose some \( \delta \leq \min \left \{ \text{excess}(v), c_e^f \right \} \) \\
  \phantom{\enskip} \phantom{\enskip} \phantom{\enskip} \( \text{push}(e,\delta) \) \enskip \textcolor{gray}{// no new steep edges} \\
  \phantom{\enskip} \phantom{\enskip} \textbf{else} \( d(v) \)++ \enskip \textcolor{gray}{// relabel; no new steep edges}
  \phantom{\enskip}
\end{pseudocode}

Wichtig ist hierbei die Invariante $\forall (v,w) \in E: d(v) \leq d(w)+1$. Ist dies nicht erfüllt, so spricht man von einer steilen (Abwärts-)Kante. 

Dieser generische Algorithmus hat eine Laufzeit von \( O(n^2m) \), die Laufzeit wird dominiert durch nonsaturating pushes.

\subsection{FIFO Preflow Push}

Bei einem Knoten werden so lange wie möglich saturierende Pushes ausgeführt bis er deaktiviert oder relabelt wird. Dann gehe zum nächsten Knoten. Die Laufzeit beträgt hierfür $O(n^{3})$. Ist daher vor allem für Graphen interessant, bei denen $m \gg n$.


\subsection{Highest Level Preflow Push}

Wählt man immer die aktiven Knoten aus, die \( d(v) \) maximieren, so kann man die Laufzeit auf \( O(n^2\sqrt{m}) \) drücken. Die Implementierung erfolgt mit Bucket PQ.

\subsection{Modified FIFO Selection Rule}
Hierbei wird ein aktivierter ausgewählter Knoten solange weiter ausgewählt, bis er inaktiv wird.

\subsection{Heuristiken}
Mit heuristischen Mitteln lässt sich der practical case weiter reduzieren. 

\begin{itemize}
	\item \textbf{Aggressive local relabeling}: Hier wird nicht inkrementell um 1 relabeled wird, sondern um den minimalen Abstand + 1 einer inzidenten Kante. \\
	\item \textbf{Global relabeling}: Zu beginng und jede $O(m)$ Kanteninspektionen werden alle Kanten mittels reverse BFS gelabelt. \\
	\item \textbf{Returning Flow}: Sonderbehandlung von Kanten mit Level größer $n$ (, da sie sowieso nur zurückfließen) \\
	\item \textbf{Gap Heuristiken}: Wenn ein Kantenlevel auf dem Pfad zu $t$ fehlt, so kann sich kein Knoten damit verbinden.
\end{itemize}
