\chapter{Randomisierte Algorithmen}

\textbf{Las Vegas Algorithmen}: Ergebnis immer korrekt, die Laufzeit ist jedoch eine Zufallsvariable. Wir kennen dafür Hashing und Quicksort.
\textbf{Monte Carlo Algorithmen}: Ergebnis bei $k$-facher Wiederholung mit Wahrscheinlichkeit $1-p^k$ korrekt. 

Einfach, effiziert und oft bestes bekanntes Verfahren. Auch algebraische Methoden können hier verwendet werden. Häufig sin randomisierte Algorithmen jedoch schwer zu analysieren. 

\section{Ergebnisüberprüfung von Sortieren}

\subsubsection{Permutationseigenschaft}
$\langle e_1, \dots , e_n \rangle$ 
ist eine Permutation von 
$\langle e_1', \dots , e_n' \rangle$ 
gdw. 
\begin{equation*}
	q(z):=\prod_{i=1}^n{(z-\text{field}(\text{key}(e_i)))} - \prod_{i=1}^n{(z-\text{field}(\text{key}(e_i')))} = 0
\end{equation*}

wobei $\mathbb{F}$ ein Körper ist, auf den alle Elemente injektiv abgebildet werden. 
Wenn dies keine Permutation ist, dann ist $q \neq 0$


Idee: Ein Polynom $n$-ten Grades hat maximal $n$ Nullstellen. Wenn wir also an einer zufälligen Stelle $x\in \mathbb{F}$ auswerten, so ist $P[q\neq 0 \wedge q(x) = 0] \leq \frac{n}{|\mathbb{F}|}$. 
Dies ist ein Monte-Carlo-Algorithmus welcher sich in Linearzeit berechnen lässt. 

Problem: Finden eines passenden Körpers $\mathbb{F}$

\subsubsection{Sort Checking}

Sei $h$ eine zufällige Hashfunktion mit Werten aus $[0,U-1]$ mit $h(S):= \sum_{e \in S}{h(e)}$.
Dann überprüfe ob die Folge $E$ eine Permutation der Folge $E'$ ist indem die beiden Hashes auf Gleichheit überprüft werden. 

Idee: Die Wahrscheinlichkeit, dass beide Folgen denselben Hashwert haben ist maximal $\frac{1}{U}$.

\section{Hashing}
Ziel: Suchen in worst case konstanter Zeit.
\subsubsection{Perfektes Hashing}
Idee: injektives $h$. Braucht allerdings $\omega(n)$ Bits Speicherplatz.

\subsubsection{Fast Space Efficient Hashing}
$n$ Elemente mit Speicherplatz $(1+\epsilon)n$ implementieren.

\subsubsection{Cuckoo Hashing}
Table mit Speicherplatz $(2+\epsilon)n$ mit zwei Hashfunktionen.
Beim Einfügen kann es zu Kollisionen kommen. Dann wird das andere Element verschoben. Diese Kettenoperationen führen mit hoher Wahrscheinlichkeit zu Erfolg. Anderenfalls wird die komplette Tabelle rebuildet. Amortisiertes Einfügen in $O(1)$. 

Kann als ein Graph modelliert werden. Knoten sind Speicherzellen. Eine ungerichtete Kante bedeutet, dass ein Element mittels der Hashfunktionen an den Endknoten gespeichert werden kann. Eine gerichtete Kante bedeutet, dass ein Element am Startknoten gespeichert wird. Jeder Knoten darf maximal eine ausgehende Kante besitzen, der Rest wird verdrängt. 

Das Einfügen funktioniert gdw. die Komponente, die die neue Kante enthält, nicht mehr Kanten als Knoten enthält. Kann gezeigt werden mittels Abstraktion in den Fall dass die Kante auf einen Tree oder einen Pseudotree zeigt.

Die Wahrscheinlichkeit für ein Rebuild liegt bei zufälligen Graphen in $O(1/n)$.




